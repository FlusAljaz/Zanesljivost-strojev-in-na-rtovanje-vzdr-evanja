\documentclass[12pt,a4paper]{article}
\usepackage{amsmath, amssymb, amsthm}
\usepackage{graphicx}
\usepackage{geometry}
\geometry{margin=2.5cm}
\usepackage{setspace}
\onehalfspacing

\title{Optimalna zamenjava stroja v prenovitvenem modelu \\ 
\large Primerjava Weibull in lognormalnega modela z degradacijo}
\author{Aljaž Flus}
\date{December 2025}

\begin{document}



\section{Uvod}

Zanesljivost strojev in načrtovanje vzdrževanja je področje, ki preučuje naključne okvare strojev, njihove posledice in optimalne strategije popravil in zamenjav. 
Temeljno vlogo imajo slučajni procesi, ki opisujejo, kako se naključni pojavi zgodijo skozi čas.

\subsection*{Slučajni procesi}

Slučajni proces je družina naključnih spremenljivk $\{X(t)\}_{t\ge 0}$, ki opisuje stanje sistema v času. 
V zanesljivosti strojev najpogosteje obravnavamo dva tipa procesov:

\begin{itemize}
    \item \textbf{proces časov med okvarami}: $T_1, T_2, \dots$, kjer je $T_i$ medprihodni časi, ki nam povejo koliko časa je minilo od $i$-tega popravila do $(i+1)$-te okvare,
    \item \textbf{proces števila okvar}: $N(t)$, število okvar do časa $t$.
\end{itemize}

Če so časi med okvarami neodvisni in enako porazdeljeni, nastane tako imenovan prenovitveni proces. 
Tak model omogoča oceno:

\begin{itemize}
    \item pričakovanega števila okvar v določenem obdobju,
    \item skupnih stroškov okvar in popravil,
    \item optimalnega časa za zamenjavo stroja.
\end{itemize}

Osnovna naloga v tej projektni nalogi je primerjava različnih modelov življenjske dobe in njihov vpliv na optimalno strategijo vzdrževanja.

\section{Modeli življenjske dobe strojev}

V tej nalogi uporabljamo tri pomembne porazdelitve časov med okvarami:

\begin{enumerate}
    \item eksponentno porazdelitev,
    \item Weibullovo porazdelitev,
    \item težkorepo - lognormalno porazdelitev.
\end{enumerate}

V nadaljevanju bomo opisali zgoraj naštete porazdelitve.

\subsection{Eksponentna porazdelitev}

Eksponentna porazdelitev se pogosto uporablja za modeliranje naključnih okvar brez staranja sistema. Verjetnost okvare je v vsakem trenutku enaka, neodvisno od starosti stroja.

Naj bo slučajna spremenljivka $T$ porazdeljena kot
\[
T \sim \mathrm{Exp}(\lambda), \qquad \lambda > 0.
\]

Porazdelitvena funkcija je
\[
F_T(t) =
\begin{cases}
1 - e^{-\lambda t}, & t \ge 0, \\
0, & t < 0.
\end{cases}
\]

Gostota verjetnosti je
\[
f_T(t) =
\begin{cases}
\lambda e^{-\lambda t}, & t \ge 0, \\
0, & t < 0.
\end{cases}
\]

Pričakovana vrednost je
\[
\mathbb{E}[T]
= \int_0^\infty t \lambda e^{-\lambda t} \, dt
= \frac{1}{\lambda}.
\]

Hazardna funkcija je konstantna, zato model ne opisuje staranja sistema.

\subsection{Weibullova porazdelitev}

Weibullova porazdelitev omogoča modeliranje staranja stroja. 

Naj bo
\[
T \sim \mathrm{Weibull}(\beta, \eta), \qquad \beta > 0, \ \eta > 0.
\]
Parameter $\beta >0$ nam pove 
kako se verjetnost okvare obnaša skozi čas. Če je vrednost parametra večja od 1 nam to pove 
da verjetnost okvare raste skozi čas. Parameter $\eta >0$ pa nam pove življensko dobo stroja, 
večji kot je, daljše bo povprečno delovanje našega stroja.

Porazdelitvena funkcija je
\[
F_T(t) =
\begin{cases}
1 - e^{-(t/\eta)^\beta}, & t \ge 0, \\
0, & t < 0.
\end{cases}
\]

Gostota verjetnosti je
\[
f_T(t) =
\begin{cases}
\displaystyle
\frac{\beta}{\eta}
\left(\frac{t}{\eta}\right)^{\beta-1}
e^{-(t/\eta)^\beta},
& t \ge 0, \\
0, & t < 0.
\end{cases}
\]

Pričakovana vrednost je
\[
\mathbb{E}[T]
= \int_0^\infty t f_T(t)\,dt
= \eta \, \Gamma\!\left(1 + \frac{1}{\beta}\right),
\]

\subsection{Lognormalna porazdelitev}

Lognormalna porazdelitev spada med težkorepe porazdelitve, katere značilnost je da njihov rep ni eksponentno omejen.
Poljudno to pomeni da se porazdelitev manjša bolj počasi kot pri eksponentni porazdelitvi. 
Zanjo so značilni težki repi, kar pomeni možnost zelo dolgih časov delovanja.

Naj bo
\[
T \sim \mathrm{Lognormal}(\mu, \sigma^2),
\qquad \mu \in \mathbb{R}, \ \sigma > 0.
\]

Porazdelitvena funkcija je
\[
F_T(t) =
\begin{cases}
\Phi\!\left(\dfrac{\ln t - \mu}{\sigma}\right), & t > 0, \\
0, & t \le 0,
\end{cases}
\]
kjer je $\Phi(\cdot)$ porazdelitvena funkcija standardne normalne porazdelitve.

Gostota verjetnosti je
\[
f_T(t) =
\begin{cases}
\displaystyle
\frac{1}{t \sigma \sqrt{2\pi}}
\exp\!\left(
-\frac{(\ln t - \mu)^2}{2\sigma^2}
\right),
& t > 0, \\
0, & t \le 0.
\end{cases}
\]

Pričakovana vrednost je
\[
\mathbb{E}[T]
= \int_0^\infty t f_T(t)\,dt
= e^{\mu + \sigma^2/2}.
\]



\newpage

\section{Pričakovano število okvar v času $[0,20]$}

Najprej bomo poiskušali izračunati število okvar analitično oziroma vsaj numerično. Tukaj bomo predpostavili 
da stroji nimajo degradacije in da takoj ko se stroj pokvari ga popravimo oziroma zamenjamo z novim.

\begin{itemize}
    \item Eksponentni model ($\lambda = 0.25)$.
    Tukaj lahko preprosto analitično izračunamo število okvar kot $E[N_T] = \lambda T$, torej je $E[N(20)] = 0.25 * 20 = 5$
    \item Weibullov model ($\beta = 2, \eta = 4$).
    Tukaj pa uporabimo znanje o prenovitvenih procesih, kjer velja formula
    $$
    E[N_T] = m(T) =F(T) + \int_{0}^{T} m(T - s)f(s) ds,
    $$
    kjer je $F$ kumilativna porazdelitvena funkcija naše slučajne spremenljivke, 
    $f$ pa njena pripadajoča gostota. Ko to izračunamo numerično dobimo $E[N(20)] = 5.2788$
    \item Lognormalni model $(\mu = ln(4) - \sigma/2, \sigma = 0.5)$.
    Tukaj je zelo važno kako določimo $\mu$, ker pri tej porazdelitvi $\mu$, kot smo ga navajeni 
    pri normalni porazdelitvi, ne pove dejansko pričakovano dobo delovanja stroja, ampka moramo ta 
    $\mu$ iz eksponentne porazdelitve tranformirat da velja $e^{\mu + \sigma / 2} = 4$.
    Tukaj prav tako uporabimo prenovitveno enačno in dobimo rezultat $E[N(20)] = 4.6424$
\end{itemize}

Sedaj pa lahko te analitično oziroma numerično izračunane rezultate primerjamo z našimi simulacijami.
Za simuliranje smo izračunali Monte Carlo simulacije, ki so numerična metoda za približke 
pričakovane vrednosti. Temelji na generiranju velikega števila realizacij slučajnega procesa in 
izračunu statističnih povprečij teh simulacij.

Za vsak model smo generirali 10000 simulacij in ocenili $E[N(20)]$ in dobili nasledje približke.

\begin{itemize}
    \item Eksponentni model ($\lambda=0.25$):
    \[
    \mathbb{E}[N(20)] \approx 5.0144.
    \]
    
    \item Weibull ($\beta=2$, $\eta=4$):
    \[
    \mathbb{E}[N(20)] \approx 5.2806.
    \]
    
    \item Lognormal ($\mu=ln(4) - \sigma / 2$, $\sigma=0.5$):
    \[
    \mathbb{E}[N(20)] \approx 4.6408,
    \]
\end{itemize}

Tukaj lahko kot komentar omenimo da opazimo da se simulacije res dobro prilegajo 
našim numeričnim oziroma analitičnim izračunom, tako da lahko rečemo da lahko z Monte Carlo simulacijami 
dobimo dober približek da dejanske izračune.

\section{Degradacija življenjske dobe stroja}

Sedaj pa bomo pogledali primer, ko pride do degradacije življenske dobe stroja ko ga popravimo. 
Predpodstavili bomo da se stroju z vsakim popravilom njegova življenjska doba zmanjša za nek odstotek.
To pa lahko zapišemo s pomočjo naslednjih enačb.
\[
\mathbb{E}[T_i] = m_0 d^{i-1},
\]
kjer je $0 < d < 1$ faktor degradacije, $m_0$ pa začetna pričakovana življenjska doba.

Za Weibullov model degradacija parametra skale $\eta$ neposredno povzroči degradacijo pričakovane vrednosti:
\[
\eta_i = \eta_0 d^{i-1}.
\]

Pri lognormalni porazdelitvi pa je potrebna previdnost. 
Če degradiramo parameter $\mu$, degradiramo median življenjske dobe, ne pa povprečja:
\[
\text{median}(T_i) = e^{\mu_i}.
\]
Da degradiramo povprečje, moramo uporabiti transformacijo:
\[
m_i = m_0 d^{i-1}, \qquad 
\mu_i = \ln(m_i) - \frac{\sigma^2}{2}.
\]
Le v tem primeru se lognormalni model obnaša usklajeno z analitično formulo za $E[T]$.

\section{Pričakovani čas cikla}

Ob predpostavki degradacije pričakovane življenjske dobe dobimo:
\[
E[\text{čas cikla do k-te okvare}]
= \sum_{i=1}^k E[T_i]
= m_0\sum_{i=0}^{k-1} d^i
= m_0 \frac{1 - d^k}{1 - d}.
\]
Ta formula nam bo omogočala analitično optimizacijo brez simulacij.

\section{Model stroškov in optimizacija}

Naj bo
\[
C_n = \text{strošek novega stroja}, \qquad
C_p = \text{strošek popravila}, \qquad
C_d = \text{izguba dohodka}.
\]
Skupni strošek cikla z $k$ popravili je:
\[
C_{\text{cikel}} = C_n + k(C_p + C_d) = C_n + kC,
\qquad C = C_p + C_d.
\]

Strošek na enoto časa je
\[
g(k) = \frac{C_n + kC}{m_0 \dfrac{1 - d^k}{1 - d}}.
\]
Mi hočemo ugotoviti kdaj je ta strošek na časovno enoto najmanjši.
Ker je $m_0/(1-d)$ konstanta, minimiziramo
\[
h(k) = \frac{C_n + kC}{1 - d^k},
\]
kjer je iskan optimalni $k \in \mathbb{N}$.

Za realni $k$ odvajamo:
\[
h'(k) = 
\frac{C(1 - d^k) - (C_n + Ck)(-\ln d)\, d^k}{(1 - d^k)^2}.
\]

Optimalna točka zadošča enačbi
\[
C(1 - d^k) + (C_n + Ck)(\ln d)\, d^k = 0.
\]
Enačbo rešimo numerično in rezultat zaokrožimo na najbližje celo število.

\section{Optimalno število popravil z našimi parametri}

Uporabili bomo naslednje parametre:
\[
C_n = 8000,\quad C_p = 600,\quad C_d = 250,\quad C = 850.
\]

Življenjske dobe:
\begin{itemize}
\item eksponentna: $\lambda = 0.25$
\item Weibull: $\beta = 2$, $\eta = 4$
\item lognormal: $\mu = ln(4) - \sigma /2$, $\sigma = 0.5$,
\end{itemize}

Degradacija: $d = 0.8$.

Za $k = 1,\dots,30$ izračunamo $g(k)$.

Tukaj dobimo analitično rešitev $k^* = 7$kar pomeni, da je optimalno opraviti $7$ popravil, preden stroj zamenjamo z novim. 
Ta rešitev je enaka za vse obravnavane porazdelitve. 
Če malo pomislimo je to zelo logična rešitev, saj nam porazdelitev določa zgolj trajanje posameznih ciklov, 
ne vplivajo pa neposredno na stroške, ki nastanejo znotraj cikla.
Ker je cilj minimizirati stroške na cikel, odločilno le število okvar, po katerik stroj zamenjamo. 
Optimalen $k^*$ je odvisen samo od stroškov modela in od njegove  degradacije.


\section{Monte Carlo simulacije}

Sedaj pa bomo analitičen izračunan $k^*=7$ preverili še če sovpada z rezultati iz Monte Carlo simulacij. 
Za vsak model smo generirali 10000 simulacij in ocenili kje dosežemo optimalen $k^*$ in dobili nasledje
približke:

\begin{itemize}
    \item Eksponentni model ($\lambda=0.25$):
    \[
    k^* = 7.
    \]
    
    \item Weibull ($\beta=2$, $\eta=4$):
    \[
    k^* = 7.
    \]
    
    \item Lognormal ($\mu=ln(4) - \sigma / 2$, $\sigma=0.5$):
    \[
    k^* = 7,
    \]
\end{itemize}

Tudi tukaj vidimo da se vrednosti lepo ujemajo z analitično izračunanimi. Torej lahko rečemo,
da res lahko s pomočjo Monte Carlo simulacijami ocenimo optimalno stevilo popravil preden stroj zamenjamo.


\section{Zaključek}

V tej projektni nalogi smo obravnavali problem optimalnega števila popravil 
preden zamenjamo stroj v okviru prenovitvenega modela z degradacijo življenjske dobe. 
Obravnavali smo tri različne porazdelitve časov med okvarami: eksponentni, Weibullov in lognormalni model.

Najprej smo za primer brez degradacije izračunali pričakovano število okvar 
v fiksnem časovnem intervalu tako analitično kot tudi numerično in te rezultate primerjali
z rezultati ki smo jih dobili z Monte Carlo simulacijami. 
Rezultati Monte Carlo simulacij so se zelo dobro ujemali z analitičnimi oziroma numeričnimi izračuni, 
kar potrjuje ustreznost simulacijskega pristopa za obravnavani problem.

V nadaljevanju smo uvedli degradacijo življenjske dobe.
Glavni rezultat naloge je analitična določitev optimalnega števila popravil $k^* = 7$, ki minimiziri dolgoročne stroške na časovno enoto, 
in primerjava njega z rezultatom ki smo ga dobili s simulacijami. Pomembno je poudariti, da je optimalna vrednost $k^*$ neodvisna od izbire 
porazdelitve časov med okvarami in je določena samo s stroškovnimi parametri in stopnjo degradacije. Izbira porazdelitve 
vpliva zgolj na trajanje posameznih ciklov, ne pa na optimalno število popravil do nakupa novega stroja.
Rezultati Monte Carlo simulacij so v vseh treh modelih potrdili analitično dobljeno optimalno rešitev, 
kar dodatno potrjuje pravilnost analitičnega rezultata in uporabnost simulacijskih metod.

\end{document}
